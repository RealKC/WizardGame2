\documentclass{article}

\usepackage[a4paper]{geometry}
\usepackage{graphicx}
\usepackage[utf8]{inputenc}
\usepackage{subcaption}
\usepackage{placeins}
\usepackage{wrapfig}
\usepackage{float}
\usepackage{minted}
\usepackage{hyperref}


\graphicspath{{./images/} {./diagrams/}}

\title{\textbf{WizardGame 2}}
\author{
    Dumitru Mițca\\
    Grupa NNNNL}
    \date{2023}

    \begin{document}
    \maketitle

    \hypersetup{linkbordercolor=1 1 1}
    \renewcommand*\contentsname{Cuprins}
    \tableofcontents
    \hypersetup{linkbordercolor=1 0 0}


    \newpage

    \section{Contextul jocului}
    WizardGame 2 este un sequel la \href{https://github.com/RealKC/WizardGame}{WizardGame}
    \footnote{Acesta a fost proiectul meu pentru POO, un sumar al poveștii veți
    regasi in \href{sec:anexa1}{Anexa 1}}.
    Dupa evenimentele \emph{WizardGame} în care Mircea l-a înfrânt pe Adrian, Mircea decide sa se
    mute la casa lui de pe Bulevardul Magheru din București, deoarece Brăila îi aduce amintiri
    triste.

    Cristian, fratele lui Adrian și necromant\footnote{Un vrăjitor care deține puteri ce
    îi permit să reînvie și să controleze morții.} renumit, care era plecat în Sicilia
    pentru a învăța vrăji de la magicienii sicilieni este anunțat de moarte fratele său
    și jură că îl va răzbuna. În următorii 2 ani, Cristian stabilizează clanul Isch'tauk,
    care devenise haotic după moartea lui Adrian, fostul lider, și se pregătește pentru
    răzbunare. Iar pe data 25 octombrie 1966 își pune în acțiune planurile.

    În dimineața acestei zile Cristian își aduce armate de nemorți, în care sunt și Matei
    și Adrian, reînviați de Cristian, și pe cei mai buni dintre discipolii lui \emph{Denis,
    the Cosmicbane} și \emph{Cezar, the Tătărașiborn Guardian} în București și astfel
    începe atacul asupra lui Mircea.

    În timpul luptei, \emph{Eduard, the Carpathian Thunder} aude lupta ce se întâmplă în jurul
    casei sale, și decide și el să lupte împotriva armatei nemoarte.

    \section{Gameplay și interacțiunea jucătorului cu jocul}

    \begin{figure}[h]
        \includegraphics[scale=0.125]{gameplayloop}
        \centering
        \caption{WizardGame 2 Gameplay loop}
    \end{figure}

    După ce jucătorul își alege un personaj \ref{sec:pcs} și intră într-un nivel, dispune
    doar de un singur atac: o vrajă specifică acelui personaj. Cât timp jucătorul se află
    în nivel, acesta va doborî inamici, care vor lăsa la poziția lor \emph{experiență}.
    Colectarea \emph{experienței} va duce eventual la creșterea nivelului jucătorului, lucru
    care îi va permite să obțină arme sau vrăji noi, sau să își îmbunătățească caracteristicile
    \ref{sec:stats}.

    Extensii asupra sistemului acesta sunt discutate după prezentarea armelor, vrăjilor și
    echipamentului \ref{sec:exts}.

    Completarea unui nivel necesită învingerea boss-ului acelui nivel. Boss-ul apare la
    minutul 14:00 și trebuie învins pentru a progresa la nivelul următor.

    Jucătorul își mișcă personajul folosing tastele W (în sus), S (în jos), A (la stânga) și D
    (la dreapta), iar abilitățile sale sunt activate automat, după un timer. Abilitățile care
    țintesc într-o direcție anume vor folosi direcția în care se mișcă jucătorul în momentul
    activării abilității. Atfel dificultatea jocului nu reiese din abilitatea jucătorului de a lovi
    inamicii, ci din abilitatea sa de a se feri de ei, sau atacurile lor, ceea ce va deveni dificil
    cu cât jucătorul se află mai mult într-un nivel.

    În fiecare nivel există obstacole prin care jucătorul nu poate trece, în plus jucătorul nu
    poate ieși din interior nivelului în afara sa.

    Jucătorul va pierde un nivel în momentul în care viața acestuia a atins valoarea 0 și va
    câștiga o data ce boss-ul nivelului este învins. Câștigarea nivelului va permite accesul la
    următorul nivel, și în anumite cazuri debloca personaje noi.

    Sistemul de save/load al jocului este automat, și nu necesită o interacțiune explicită a
    jucătorului.

    \section{Conținut}
    \label{sec:content}

    \emph{WizardGame 2} conține o varietate de inamici ce trebuie învinși de
    jucător și o varietate de vrăji și atacuri ce pot fi folosite de acesta pentru a ajunge
    la victorie.

    \label{sec:sprites}
    \begin{figure}[h]
        \includegraphics[scale=2]{player-characters}
        \centering
        \caption{Protagoniștii, de la stânga la dreapta: Mircea, Eduard, Mihai}
    \end{figure}

    Jucătorul poate alege între trei personaje: \emph{Mircea, The Wizard of Brăila},
    \emph{Mihai, the Fullmetal Wizard} și \emph{Eduard, the Carpathian Thunder}.
    \begin{itemize}
        \item \emph{Mircea} --- începe cu o vrajă care îi permite să arunce bile de foc
        \item \emph{Mihai} --- deoarece și-a pierdut brațul în Războiul Magilor din 1959,
        și-a implatat un braț magic metalic. Acesta îl face să fie mai slab cu atacuri
        magice, dar mai priceput cu atacuri ce presupun aruncarea unui obiect.
        \item \emph{Eduard} --- un vrăjitor cu control asupra fulgerului, astfel atacul
        său este un fulger. Jucătorul deblochează accesul la acest personaj după ce
        învinge primul nivel. Datorită reflexelor sale excelente, el are șansa atacului critic crescută.
    \end{itemize}

    Jucătorul poate folosi maxim 3 din urmatoarele vrăji și arme:
    \begin{itemize}
        \item \emph{Defendere Magi} --- o carte de vrăji care creează o zonă circulară în jurul
        jucătorului care rănește inamicii.
        \item \emph{Circulus Glaciei} --- o carte de vrăji care creează un cerc de țurțuri în jurul
        jucătorului.
        \item \emph{Ignis Respiratio} --- o carte de vrăji care creează o flacară în direcția în
        care se mișcă jucătorul.
        \item \emph{Terra Gutta} --- o carte de vrăji care creează blocuri de pământ în locații
        aleatoare pe hartă.
        \item \emph{Vortex Mercurius} --- o carte de vrăji care creează tornade ce pleacă de la
        jucător într-o direcție aleasă aleator.
        \item \emph{Pistol Carpați} --- un pistol, viteza sa de atac este crescută comparativ cu
        folosirea unei vrăji.
        \item \emph{Baltagul Vitoriei} --- un topor gigant, care pleacă în spirală de la jucător.
        spre exterior.
        \item \emph{Morning Star} --- o bila metalică cu țepi atașată cu un lanț de un mâner, se
        rotește în jurul jucătorului.
        \item \emph{Holy Lamp} --- crează o arie conică în fața jucătorului care are bonus de putere
        împotriva inamicilor nemorți.
    \end{itemize}

    Jucătorul poate echipa maxim 3 din următoarele seturi de echipament, care îi vor conferi
    efecte pasive:
    \begin{itemize}
        \item \emph{Fluierul Sfintei Vineri} --- oferă un bonus vitezei jucătorului.
        \item \emph{Colier cu Safir} --- oferă un bonus puterii magicii jucătorului.
        \item \emph{Glasses of Infirma Aspectu} --- crește șansa atacului critic.
        \item \emph{Cape of Great Magic} --- crește puterea magică a jucatorului, dar scade viteza
        de atac.
        \item \emph{Orichalcum Magnet} --- crește raza de pickup a jucătorului.
        \item \emph{Mănușile Alchimistului Focului} --- oferă un bonus puterii magicii jucătorului,
        crescut față de \emph{Colierul cu Safir}, dar va scădea raza de pickup. Bonusul v-a fi mai
        mare când sunt folosite de \emph{Mircea}.
        \item \emph{Chainmail Shirt} --- permite jucătorului să supraviețuiască mai multor atacuri
        înainte de a muri.
        \item \emph{Ring of Anima} --- permite jucătorului să-și regeneze viața pierdută.
        \item \emph{Cercei cu Rubine} --- crește viteza de atac a jucătorului.
    \end{itemize}

    Personajul jucătorului are următoarele caracteristici, care pot fi îmbunătățite pe parcursul
    jocului:
    \begin{itemize} \label{sec:stats}
        \item viața (\textbf{HP}) --- se referă la câte atacuri poate jucătorul suferi înainte de a muri.
        \item puterea magică (\textbf{+ATK\%}) --- un bonus procentual aplicat asupra unui atac normal.
        \item șansa atacului critic (\textbf{CRIT\%}) --- șansa ca un atac să își aibă valoare dublată.
        \item viteza (\textbf{+SPD\%}) --- cât de rapid se mișca personajul jucătorului.
        \item raza de pickup (\textbf{+PCK}) --- raza maximă în care trebuie să se afle un item
        pe hartă pentru a putea fi luat de jucător, măsurată în pixeli.
        \item viteza de atac (\textbf{+HST\%}) --- cât de rapid atacă jucătorul.
    \end{itemize}
    \label{sec:pcs}

    Cele enumerate mai sus ar fi obținute de jucător în momentul în care și-a crescut nivelul,
    fiindu-i puse la dispoziție o alegere între arme, vrăji, echipament și posibilitatea de a
    crește valoarea unei caracteristici.

    \label{sec:exts}
    Jucătorul va trebuie să lupte cu inamici simpli (zombie, scheleți, fantome), cu inamici de tip
    mini-boss, care sunt doar versiuni mai mari și mai puternici a celor simpli, și cu inamici de
    tip boss, care sunt următorii și care sunt importanți poveștii jocului:
    \begin{itemize}
        \item \emph{Matei, the Prophaned} --- sensei-ul lui Mircea, a murit în \emph{WizardGame}, dar a
        fost reînviat de Cristian pentru a ajuta în răzbunarea lui. Puțin rămâne din shamanul care
        a fost Matei în viață. Matei folosește magia umbrelor, învățată de el în timpul Războiului Magilor.
        \item \emph{Adrian, the Crimson Undead} --- fostul lider al clanului Isch'tauk și fratele
        lui Cristian. Se pare că doliul l-a dus pe Cristian să facă acte de neiertat propriei familii.
        \item \emph{Cezar, the Tătărașiborn Guardian} --- unul dintre cei mai buni discipoli ai lui
        Cristian, născut în Tătărași.
        \item \emph{Denis, the Cosmicbane} --- unul dintre cei mai buni discipoli ai lui Cristian, și
        mag cu puteri spațiale, specialitățile lui sunt teleportarea și invizibilitatea.
        \item \emph{Cristian, the Necromancer of Neamț} --- caută să își răzbune fratele. Jucătorul va
        lupta cu el în două faze.
    \end{itemize}


    \section{Niveluri}

    \begin{figure}[h]
        \includegraphics[width=0.55\textwidth, angle=90]{designing-levels}
        \centering
        \caption{Proiectarea nivelurilor}
    \end{figure}

    \emph{WizardGame 2} are următoarele niveluri:
    \begin{itemize}
        \item \emph{Curtea lui Mircea} --- un cămp larg, în care obstacolele sunt garduri,
        pietre și copaci, dar care în general oferă o libertate de mișcare crescută. Inamicii
        prezenți vor fi schelete, cranii zburătoare și zombie. Inamicii de tip boss vor fi
        \emph{Matei, the Profaned} și \emph{Adrian, the Crimson Undead}.
        \item \emph{Bulevardul Magheru} --- un nivel mai îngust, în care obstacolele vor fi
        mașini parcate. Inamicii prezenți vor fi schelete cu armură, zombie și fantome. Inamicii
        de tip boss vor fi \emph{Denis, the Cosmicbane} și \emph{Cezar, the Tătărașiborn Guardian}.
        \item \emph{Calea Victoriei} --- un nivel și mai îngust care va încapea în totalitate pe
        ecran în înălțime, dar nu în lățime. \emph{Cristian, the Necromancer of Neamț} va fi
        singurul inamic de tip boss, dar lupta cu el va fi în două etape.
    \end{itemize}

    În toate aceste nivele vor exista inamici de tip mini-boss care sunt versiuni mai mari și mai
    puternice a celor normali.

    Dificultatea va crește în fiecare nivel prin incluziunea inamicilor care apar după jumătatea
    timpului în nivelul anterior și prin creșterea puterii ataculurilor inamicilor.

    \section{Interfața grafică}

    \begin{figure}[H]
        \includegraphics[width=0.45\textwidth, angle=90]{designing-menus}
        \centering
        \caption{Proiectarea meniurilor aplicației}
    \end{figure}

    Meniurile jocului sunt standard și proiectate pentru a fi intuitive pentru oricine a mai jucat
    jocuri. Interacțiunea cu elementele meniurilor se va realiza folosing cursorul mouse-ului, dar
    va fi permisă și folosirea tastelor, ca și în restul aplicațiilor grafice.

    \begin{figure}[H]
        \includegraphics[width=0.45\textwidth, angle=90]{designing-level-ui}
        \centering
        \caption{Schița interfeței grafice într-un nivel}
    \end{figure}

    Interfața grafică dintr-un nivel a fost proiectată cu scopul de a nu distrage de
    la acțiunea principală, dar în același timp ea oferă informațiile esențiale de care are
    nevoie jucătorul.

    În general am urmat simplicitatea în proiectarea interfeței grafice, pentru a nu distrage
    jucătorul de la ce este important.

    \section{Algoritmi utilizați}

    \subsection{Coliziuni}
    \begin{figure}[H]
        \includegraphics[width=\linewidth]{collision.png}
        \caption{Două obiecte care se intersectează si proiecțile lor pe axe. Copyright Lazy Foo' Productions 2004-2023}
        \centering
    \end{figure}

    Coliziunile sunt detectate folosind algoritmul \emph{AABB(Axis-Aligned Bounding Box) Collision},
    care proiectează dreptunghiuri pe segmente și verifică dacă acestea se intersecteaza

    \subsection{Mișcarea inamicilor}

    Mișcarea inamicilor este în general implementat urmărind jucătorul exact și incercându-se
    apropierea de acesta. Pentru inamicii de tip boss, aceasta este mai complicată și poate
    implica: generarea unei poziții aleatoare apropiate de jucător către care boss-ul se va deplasa,
    sau generea unei poziții aleatoare în jurul jucătorului la care boss-ul se va teleporta.

    \subsection{Instanțierea nivelelor}

    Când un nivel este instanțiat, acesta primește o listă de dale cărora le este asignată o cheie,
    un șablon (de exemplu \texttt{MMM\textbackslash nMMM\textbackslash nMMM}), o listă de obstacole
    și un dicționar care mapează poziții pe o dală la un anumit obstacol.

    Instanțierea nivelului lipește dalele împreună urmând șablonul și apoi plasează obstacolele în
    lume bazându-se pe dicționarul primit ca parametru într-un mod repetat pentru fiecare dală.

    \subsection{Crearea inamicilor}

    Pentru crearea inamicilor, un număr aleator de inamici este ales, apoi o poziție apropiată de
    jucător este aleasă pentru fiecare inamic; Într-un final, toți inamicii generați sunt adăugați
    în lume.


    \section{Architectural Design Document}
    \FloatBarrier
    \begin{figure}[H]
        \includegraphics[width=\linewidth]{high-level-diagram}
        \centering
        \caption{Diagrama de nivel înalt a jocului}
    \end{figure}

    Jocul este compus din diferite clase care implementează diferite părți ale logicii sale. Astfel,
    jocul poate fi separat în următoarele componente:
    \begin{itemize}
        \item Clasele din pachetul \texttt{GameObjects} implementează logica obiectelor cu care
        jucătorul poate interacționa fizic într-un nivel.
        \item Clasele din pachetul \texttt{Items} reprezintă armele, vrăjile și echipamentul
        jucătorului.
        \item Clasele din pachetul \texttt{Scenes} care implementează logica diferitelor scene ale
        jocului (meniuri, nivelul în sine).
        \item Clasa \texttt{Game} care aduce toate clasele împreună și folosind șablonul
        \emph{main loop} realizează interactivitatea cu jucătorul.
    \end{itemize}

    \subsection{Pachetul \texttt{GameObjects}}
    \begin{figure}[H]
        \includegraphics[width=\linewidth]{gameobjects-diagram}
        \centering
        \caption{Diagrama UML a game object-urilor}
    \end{figure}

    Subclasele clasei \texttt{GameObject} reprezintă obiecte care au interacțiuni cu alte obiecte
    ale jocului: personajul jucătorului în sine, inamicii, proiectile și obstacole.

    Un \texttt{GameObject} se poate ciocni cu alte \texttt{GameObject}-e, procedeu guvernat de
    \emph{AABB Collision}\footnote{Acest algoritm verifică dacă două dreptunghiuri se intersectează,
    dreptunghiurile fiind definite de proprietățile \texttt{x}, \texttt{y}, \texttt{hitboxWidth} și
    \texttt{hitboxHeight}.}

    Notabile sunt metodele \texttt{update} și \texttt{render} deoarece sunt apelate în fiecare frame
    pentru toate \texttt{GameObject}-urile. Prima metodă actualizează statusul unui anumit obiect în
    funcție de schimbările stării jocului, iar a doua desenează pe ecran obiectul.

    Interfața \texttt{Player.PositionObserver} permite unui alt \texttt{GameObject} (și nu numai) să
    se aboneze la schimbările poziției jucătorului, primind notificări de fiecare dată când poziția
    jucătorului se schimbă. O astfel de notificare include noua poziție și unghiul în care se mișcă
    jucătorul.

    Clasele \texttt{Obstacle} și \texttt{Enemy} au proprietăți definite în fișiere JSON. Formatul
    unui obstacol este următorul:

    \begin{minted}[linenos, breaklines]{json}
    {
        "identificator": {
            "x": "Coloana din spritesheet în care se află acest obstacol",
            "y": "Rândul din spritesheet în care se află acest obstacol"
        }
    }
    \end{minted}

    Iar formatul unui inamic este:
    \begin{minted}[linenos, breaklines]{json}
        {
            "name": "identificator",
            "health": "Valoarea vieții acestui inamic",
            "x": "Coloana din spritesheet în care se află acest inamic",
            "y": "Rândul din spritesheet în care se află acest inamic"
        }
    \end{minted}

    \subsection{Pachetul \texttt{Items}}
    \begin{figure}[H]
        \includegraphics[width=\linewidth]{items-diagram}
        \centering
        \caption{Diagrama UML a item-urilor}
    \end{figure}

    Subclasele clasei \texttt{Item} reprezintă armele, vrăjile și seturile de echipament menționate
    în secțiunea \ref{sec:content}. Aceste clase nu sunt instanțiate direct, ci prin subclase ale
    clasei \texttt{ItemFactory} care abstractizează modul de creare a unui \texttt{Item} (care
    poate fi diferit în funcție de subclasă).

    Clasa \texttt{Item} implementează și interfața \texttt{Player.PositionObserver}, care îi permite
    să primească notificări legate de schimbare poziției jucătorului în lume, prin metoda
    \texttt{notifyAboutNewPosition} care primește noile coordonate ale jucătorului și unghiul în
    care se mișcă acesta.

    Notabil este faptul că item-urile nu sunt definite în cod, ci printr-un fișier JSON al cărui
    format este următorul format:

    \label{sec:item-json}
    \begin{minted}[linenos, breaklines]{json}
        {
            "name": "Numele item-ului",
            "baseAttack": "Valoarea scăzută din viața unui inamic lovit de atac",
            "speed": "Valoarea care controlează activarea item-ului",
            "x": "Coloana din spritesheet în care se află acest item",
            "y": "Rândul din spritesheet în care se află acest item",
            "itemFactoryName": "Numele complet al clasei factory care poate creea instanțe de acest item"
        }
    \end{minted}

    \subsection{Clasa \texttt{Level}}
    \begin{figure}[H]
        \includegraphics[width=\textwidth]{level-diagram}
        \centering
        \caption{Diagrama UML a clasei \texttt{Level}}
    \end{figure}

    Clasa \texttt{Level} implementează funcționalitatea unui nivel, utilizându-se de clasele
    \texttt{LevelData}, \texttt{Wave} și \texttt{EnemyDistribution}. Nivele sunt definite în fișiere
    JSON folosind următorul format:

    \inputminted[linenos, breaklines]{json}{example-level.json}

    \subsection{Șabloane de proiectare}

    Pentru a asigura o arhitectură extensibilă am folosit diferite șabloane de proiectare.

    \subsubsection{Singleton}
    \begin{figure}[H]
        \includegraphics[width=\textwidth]{singletons-diagram}
        \centering
        \caption{Două exemple de Singleton din joc}
    \end{figure}

    Singleton este cel mai utilizat șablon de proiectare din proiect, datorită flexibilității
    oferite de acesta, enumăr următoarele clase care folosesc acest șablon: \texttt{Game},
    \texttt{LevelScene}, \texttt{OSTManager}, \texttt{Assets} și \texttt{DatabaseManager}.

    \subsubsection{Builder}
    \begin{figure}[H]
        \includegraphics[width=\textwidth]{builder-diagram}
        \centering
        \caption{Diagrama care prezintă utilizarea șablonului \emph{Builder} în joc}
    \end{figure}

    Datorită complexității creării instanțelor \texttt{Enemy}, am folosit șablonul \emph{Builder}
    pentru a encapsula acest proces și pentru a expune o interfață ușor de folosit.

    \subsubsection{Factory method}
    \begin{figure}[H]
        \includegraphics[width=\textwidth]{factory-diagram}
        \centering
        \caption{Clasele de bază pentru ierharhia itemurilor}
    \end{figure}

    Datorită numărului crescut de iteme din joc, și pentru a face procesul adăugării itemelor în joc
    ușor, am folosit acest șablon de proiectare.

    \texttt{ItemData} este o clasă care conține datele aferente unui item, și este definită în
    fișiere JSON descrise anterior \ref{sec:item-json}.

    \subsubsection{Observer}
    \begin{figure}[H]
        \includegraphics[width=\textwidth]{observers-diagram}
        \centering
        \caption{Interfețele \emph{Observer} pe care le expune clasa \texttt{Player}}
    \end{figure}

    Clasa \texttt{Player} este o clasă destul de interesantă din punctul de vedere al scenelor și
    inamicilor.

    Clasa \texttt{LevelScene} implementează interfața \texttt{LevelUpObserver} și se adaugă ca
    observer când instațiază clasa \texttt{Player}. În momentul în care aceasta primește o notificare
    că jucătorul a avansat în nivel, ea notifică \texttt{SceneManager} că va trebui sa treacă într-o
    scenă nouă, și anume \texttt{LevelScene}, unde jucătorul poate alege iteme sau creșterea unei
    caracteristice proprii.

    Intefața \texttt{PositionObserver} este folosită de inamicii din joc pentru a ști unde se află
    jucătorul și naviga spre el. Totodata, armele jucătorului implementează această interfață pentru
    a ști unde trebuie plasate gloanțele lor.

    \subsubsection{State}

    \begin{figure}[H]
        \includegraphics[width=\textwidth]{state-diagram}
        \centering
        \caption{Diagrama de stări a jocului}
    \end{figure}

    În cod, stările acestea sunt implementate în pachetul \texttt{Scenes}, cu o clasă pentru fiecare
    stare (pentru starea \emph{State}, clasa se numește \texttt{\emph{State}Scene}). Toate aceste
    clase implementează următoare interfață:

    \inputminted[linenos, breaklines]{java}{Scene.java}

    Pentru realizarea tranzițiilor între stări, este folosită o clasă \texttt{SceneManager} al cărei
    rol este simplu:
    \begin{itemize}
        \item când o scenă cere progres către scena următoare, \texttt{SceneManager} cere stării
        instanța stării următoare
        \item apelează funcția \texttt{render} a scenei pentru fiecare frame
    \end{itemize}

    \subsubsection{Adapter}

    \subsection{Baza de date}
    \begin{figure}[H]
        \includegraphics[width=\textwidth]{er-diagram}
        \centering
        \caption{Diagrama E-R a bazei de date}
    \end{figure}

    Jocul folosește o bază de date \texttt{sqlite3}, stocată în directorul în care rulează jocul
    numită \texttt{saves.db}.

    Baza de date conține trei tabele:
    \begin{itemize}
        \item \emph{characters} --- în această tabelă se stochează numele personajelor și dacă acel
        personaj a fost deblocat sau nu
        \item \emph{scores} --- în această tabelă se stochează scorurile jucătorului pentru un anumit
        nivel și când au fost obținute
        \item \emph{lastFinishedLevel} --- în această tabelă se stochează ultimul nivel completat
        dintr-o campanie de nivele. Există o singură campanie de nivele în joc.
    \end{itemize}

    Datorită simplității datelor stocate în baza de date, nu există relații între tabele

    \section{Galerie}

    \addcontentsline{toc}{section}{Anexa 1}
    \section*{Anexa 1}
    \label{sec:anexa1}
    Evenimentele \emph{WizardGame} se întâmplă cu 2 ani înainte de \emph{WizardGame 2}, în 1964,
    când \emph{Adrian, the Shaman of Neamț} anunță că va distruge Munții Carpați după ce tocmai
    a furat toate cărțile de vrăji ale vrăjitorilor români. Mircea, cunoscând vraja \emph{Fireball}
    în foarte multe detalii nu are nevoie de cărți magice pentru a folosi vraja, și decide să îl
    oprească pe Adrian, dar \emph{Matei}, senseiul său este doborât de mionionii lui Adrian.
    Jocul se termină prin moartea lui Adrian și salvarea Munților Carpați.

    \addcontentsline{toc}{section}{Anexa 2}
    \section*{Anexa 2}
    Pe parcursul dezvoltării \emph{WizardGame 2} am avut diferite idei care nu s-au materializat,
    dar doresc să le menționez:
    \begin{itemize}
        \item utilizarea unui navmesh\footnote{\href{https://en.wikipedia.org/wiki/Navigation_mesh}{Navigation Mesh - Wikipedia}}
        pentru navigația inamicilor în lume către jucător, dar am ajuns la concluzia că ar fi adus
        prea multă complexitate în joc
        \item prezența unui sistem de îmbunătățire a echipamentului jucătorului temporar (pe durata
        unei runde de joc) sau permanent (folosind un sistem de \emph{currency} și un magazin extern
        nivelelor)
        \item un sistem de \emph{save slots} care să îi permită jucătorului să înceapă jocul de la
        început fără să își piardă permanent progresul
    \end{itemize}

    \addcontentsline{toc}{section}{Bibliografie}
    \section*{Bibliografie}
    Jocul este inspirat conceptual și mecanic de jocurile \emph{Vampire Survivors}
    \href{https://store.steampowered.com/app/1794680/Vampire_Survivors/}{Steam} și
    \emph{Holocure} \href{https://kay-yu.itch.io/holocure}{itch.io}.

    Sprite-urile de pe pagina \pageref{sec:sprites} sunt realizate folosind tile-urile publicate
    pe următoare pagină web: \url{https://opengameart.org/content/dungeon-crawl-32x32-tiles}.

    Pentru o listă exhaustivă care indică autorii tuturor sprite-urilor din joc, vă rog urmați
    acest link: \url{https://github.com/RealKC/WizardGame2/blob/master/res/textures/README.md}.

    Muzica jocului este realizată de minunatul meu coleg Adi, căruia îi mulțumesc.

\end{document}
