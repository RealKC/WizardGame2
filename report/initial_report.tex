\documentclass{article}

\usepackage[a4paper]{geometry}
\usepackage{graphicx}
\usepackage[utf8]{inputenc}
\usepackage{subcaption}
\usepackage{hyperref}


\graphicspath{{./images/}}

\title{\textbf{WizardGame 2}}
\author{
    Dumitru Mițca\\
    Grupa NNNNL}
    \date{2023}

    \begin{document}
    \maketitle

    \hypersetup{linkbordercolor=1 1 1}
    \renewcommand*\contentsname{Cuprins}
    \tableofcontents
    \hypersetup{linkbordercolor=1 0 0}


    \newpage

    \section{Contextul jocului}
    WizardGame 2 este un sequel la \href{https://github.com/RealKC/WizardGame}{WizardGame}
    \footnote{Acesta a fost proiectul meu pentru POO, un sumar al povestii veti
    regasi in \href{sec:anexa1}{Anexa 1}}.
    Dupa evenimentele \emph{WizardGame}, Mircea decide sa se mute la casa lui de pe
    Bulevardul Magheru din București, deoarece Brăila îi aduce amintiri triste.

    Cristian, fratele lui Adrian și necromant\footnote{Un vrăjitor care deține puteri ce
    îi permit să reînvie și să controleze morții.} renumit, care era plecat în Sicilia
    pentru a învăța vrăji de la magicienii sicilieni este anunțat de moarte fratele său
    și jură că îl va răzbuna. În următorii 2 ani, Cristian stabilizează clanul Isch'tauk,
    care devenise haotic după moartea lui Adrian, fostul lider, și se pregătește pentru
    răzbunare. Iar pe data 25 octombrie 1966 își pune în acțiune planurile.

    în dimineața acestei zile Cristian își aduce armate de nemorți, în care sunt și Matei
    și Adrian, reînviați de Cristian, și pe cei mai buni dintre discipolii lui \emph{Denis}
    și \emph{Cezar, the Tatarasiborn Guardian} în București și astfel începe atacul asupra
    lui Mircea.

    \section{Gameplay}

    \begin{figure}[h]
        \includegraphics[scale=0.125]{gameplayloop}
        \centering
        \caption{WizardGame 2 Gameplay loop}
    \end{figure}

    După ce jucătorul își alege un personaj \ref{sec:pcs} și intră într-un nivel, dispune
    doar de un singur atac: o vrajă specifică acelui personaj. Cât timp jucătorul se află
    în nivel, acesta va omorî inamici, care vor lăsa la poziția lor \emph{experiență}.
    Colectarea \emph{experienței} va duce eventual la creșterea nivelului jucătorului, lucru
    care îi va permite să obțină arme sau vrăji noi, sau să își îmbunătățească caracteristicile
    \ref{sec:stats}.

    O posibilă extensie asupra acestui sistem ar reprezenta-o existența unui mod de a crește
    abilitățile intrinsice în mod permanent. Acesta ar implica existența unui sistem de puncte
    care pot fi folosite în acest scop. Momentan nu sunt sigur că acest lucru va fi prezent
    în versiunea finală a jocului, dar îl voi explora.

    Completarea unui nivel necesită învingerea boss-ului acelui nivel. Boss-ul apare la
    minutul 20:00\footnote{Dacă pe parcursul procesului de dezvoltare observ că 20 de
    minute este prea mult pentru un nivel, timpul va fi scurtat, dar nu la mai puțin de
    14 minute.} și trebuie învins pentru a progresa la nivelul următor.

    \section{Conținut}

    \emph{WizardGame 2} conține o varietate de inamici ce trebuie învinși de
    jucător și o varietate de vrăji și atacuri ce pot fi folosite de acesta pentru a ajunge
    la victorie.

    Jucătorul poate alege între două personaje: \emph{Mircea, The Wizard of Brăila} și
    \emph{Mihai, the Fullmetal Wizard}.
    \begin{itemize}
        \item \emph{Mircea} --- începe cu o vrajă care îi permite să arunce bile de foc
        \item \emph{Mihai} --- deoarece și-a pierdut brațul în Războiul Magilor din 1959,
        și-a implatat un braț magic metalic. Acesta îl face să fie mai slab cu atacuri
        magice, dar mai priceput cu atacuri ce presupun aruncarea unui obiect.
    \end{itemize}

    Jucătorul poate folosi maxim 3 din urmatoarele vrăji și atacuri:
    \begin{itemize}
        \item TODO
    \end{itemize}

    Jucătorul poate echipa maxim 3 din următoarele seturi de echipament, care îi vor conferi
    efecte pasive:
    \begin{itemize}
        \item TODO
    \end{itemize}

    Personajul jucătorului are următoarele caracteristici, care pot fi îmbunătățite pe parcursul
    jocului:
    \begin{itemize} \label{sec:stats}
        \item puterea magică (\textbf{+ATK\%}) --- un bonus procentual aplicat asupra unui atac normal.
        \item șansa atacului critic (\textbf{CRIT\%}) --- șansa ca un atac să își aibă valoare dublată.
        \item viteza (\textbf{+SPD\%}) --- cât de rapid se mișca personajul jucătorului.
        \item raza de pickup (\textbf{+PCK}) --- raza maximă în care trebuie să se afle un item
        pe hartă pentru a putea fi luat de jucător, măsurată în pixeli.
        \item viteza de atac (\textbf{+HST\%}) --- cât de rapid atacă jucătorul.
    \end{itemize}
    \label{sec:pcs}

    \section{Niveluri}

    \begin{figure}[h]
        % \includegraphics[scale=0.125]{designing-levels}
        \centering
        \caption{Proiectarea nivelurilor}
    \end{figure}

    \emph{WizardGame 2} are următoarele niveluri:
    \begin{itemize}
        \item \emph{Curtea lui Mircea} --- un cămp larg, în care obstacolele sunt garduri,
        pietre și copaci, dar care în general oferă o libertate de mișcare crescută. Inamicii
        prezenți vor fi schelete, cranii zburătoare și zombie. Inamicii de tip boss vor fi
        \emph{Matei, the Profaned} și \emph{Adrian, the Crimson Undead}.
        \item \emph{Bulevardul Magheru} --- un nivel mai îngust, în care obstacolele vor fi
        mașini parcate. Inamicii prezenți vor fi schelete cu armură, zombie și fantome. Inamicii
        de tip boss vor fi \emph{Denis} și \emph{Cezar, the Tătărașiborn Guardian}.
        \item \emph{Calea Victoriei} --- un nivel și mai îngust care va încapea în totalitate pe
        ecran în înălțime, dar nu în lățime. \emph{Cristian, the Necromancer of Neamț} va fi
        singurul inamic de tip boss, dar lupta cu el va fi în două etape.
    \end{itemize}

    în toate aceste nivele vor exista inamici de tip mini-boss care sunt versiuni mai mari și mai
    puternice a celor normali.

    \section{Interfața grafică}

    TODO

    \addcontentsline{toc}{section}{Anexa 1}
    \section*{Anexa 1}
    Evenimentele \emph{WizardGame} se întâmplă cu 2 ani înainte de \emph{WizardGame 2}, în 1964,
    când \emph{Adrian, the Shaman of Neamț} anunță că va distruge Munții Carpați după ce tocmai
    a furat toate cărțile de vrăji ale vrăjitorilor români. Mircea, cunoscând vraja \emph{Fireball}
    în foarte multe detalii nu are nevoie de cărți magice pentru a folosi vraja, și decide să îl
    oprească pe Adrian, dar \emph{Matei}, senseiul său este omorât de mionionii lui Adrian.
    Jocul se termină prin moartea lui Adrian și salvarea Munților Carpați.

    \label{sec:anexa1}

\end{document}
